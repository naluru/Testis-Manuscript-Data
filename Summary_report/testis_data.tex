\documentclass{article}
% General document formatting
\usepackage[margin=0.7in]{geometry}
\usepackage[parfill]{parskip}
\usepackage[utf8]{inputenc}
\usepackage{color}

% Related to math
\usepackage{amsmath,amssymb,amsfonts,amsthm}

\newcommand{\TODO}[1]{\begingroup\color{red}Todo: #1\endgroup}

\begin{document}

\section{Results}

\begin{table}
  \begin{tabular}{ll|rr|rrr}
         &         &  \multicolumn{2}{c}{Number of reads} & \multicolumn{3}{c}{Number of mapped reads} \\
         & Sample  & Original  & Trimmed            & unique paired end & multiple paired end & sum of mapped reads \\
  \hline                                          
  DMSO   & D1      &  8,168,404 &  8,152,495        & 5,254,278 (64.45\%) & 1,920,049 (23.55\%) &  7,632,045 (93.62\%)   \\
  DMSO   & D2      &  9,957,843 &  9,939,550        & 6,796,802 (68.38\%) & 2,066,186 (20.79\%) &  9,412,813 (94.70\%)   \\
  DMSO   & D3      & 10,786,595 & 10,768,858        & 7,323,091 (68.00\%)    & 2,314,672 (21.49\%) & 10,233,729 (95.03\%) \\
  DMSO   & D4      & 11,799,398 & 11,777,525        & 8,055,836 (68.40\%)  & 2,496,782 (21.20\%)  & 11,191,119 (95.02\%) \\
  DMSO   & D5      & 14,501,290 & 14,457,774        & 9,602,482 (66.42\%) & 3,175,671 (21.97\%) & 13,578,935 (93.92\%) \\
    \hline
  PCB 0.3nM & P0.3-9  & 13,635,225 & 13,612,736        & 8,937,657 (65.66\%) & 3,112,057 (22.86\%) & 12,757,987 (93.72\%) \\
  PCB 0.3nM & P0.3-11 & 13,674,864 & 13,640,724        & 9,334,595 (68.43\%) & 2,863,975 (21.00\%)    & 12,977,478 (95.14\%) \\
  PCB 0.3nM & P0.3-12 & 11,830,503 & 11,806,900        & 8,038,530 (68.08\%) & 2,579,996 (21.85\%) & 11,269,978 (95.45\%) \\
  PCB 0.3nM & P0.3-13 & 13,968,933 & 13,939,004        & 9,631,230 (69.10\%)  & 2,831,576 (20.31\%) & 13,279,248 (95.27\%) \\
  PCB 0.3nM & P0.3-14 & 12,536,363 & 12,507,078        & 8,589,984 (68.68\%) & 2,552,200 (20.41\%) & 11,890,055 (95.07\%) \\
  PCB 0.3nM & P0.3-16 & 12,210,513 & 12,189,670        & 8,214,909 (67.39\%) & 2,711,764 (22.25\%) & 11,558,361 (94.82\%) \\
    \hline
  PCB 10nM & P10-17  & 11,824,432 & 11,802,545        & 8,155,563 (69.10\%)  & 2,421,282 (20.51\%) & 11,224,550 (95.10\%)  \\
  PCB 10nM & P10-18  &  9,949,807 &  9,923,567        & 6,819,261 (68.72\%) & 2,022,738 (20.38\%) &  9,424,059 (94.97\%)  \\
  PCB 10nM & P10-21  & 10,983,355 & 10,948,290        & 7,531,889 (68.80\%)  & 2,226,969 (20.34\%) & 10,403,421 (95.02\%) \\
  PCB 10nM & P10-23  & 12,880,370 & 12,854,923        & 8,907,689 (69.29\%) & 2,591,122 (20.16\%) & 12,242,973 (95.24\%) \\
  PCB 10nM & P10-24  & 11,786,444 & 11,759,002        & 8,215,663 (69.87\%) & 2,306,159 (19.61\%) & 11,224,951 (95.46\%) \\
    \hline
  Average & &        11,905,896 & 11,880,040        & 8,088,091 (68.08\%) & 2,512,075 (21.15\%) & 11,268,856 (94.86\%) \\
  \end{tabular}
  \caption{\#reads - one read consists of two pairs or fragments;
    unique/multiple paired end means both fragments are mapped either
    unique or multiple; ``sum of mapped reads'' includes additional
    options, e.g. only one fragment is unique/multiple but not the
    other; the percentages in ``Number of mapped reads'' are in
    relation to the number of trimmed reads;}
    \label{tab:mapping-stats}
\end{table}

\begin{table}
  \begin{tabular}{ll|rrrl}
       & Sample  & \#Cytosines   & \#mCpG & \#un-mCpG & Global CpG methyl.level \\   
\hline                                                                               
DMSO   & D1      &    26,250,848 &  1,531,846  &  174,245 & 73.28\%  \\              
DMSO   & D2      &    29,817,798 &  1,785,444  &  229,466 & 71.57\%  \\              
DMSO   & D3      &    29,616,929 &  1,877,577  &  247,569 & 71.41\%  \\              
DMSO   & D4      &    29,830,337 &  1,966,073  &  275,366 & 70.68\%  \\              
DMSO   & D5      &    31,389,808 &  2,251,946  &  316,096 & 71.20\%  \\              
  \hline                                                                             
PCB 0.3nM & P0.3-9  &    31,312,106 &  2,141,677    &  279,279  & 71.28\%  \\           
PCB 0.3nM & P0.3-11 &    30,940,958 &  2,129,553    &  297,305  & 71.17\%  \\           
PCB 0.3nM & P0.3-12 &    41,028,708 &  1,937,522    &  228,690  & 67.79\%  \\           
PCB 0.3nM & P0.3-13 &    31,374,908 &  2,146,969    &  299,559  & 71.14\%  \\           
PCB 0.3nM & P0.3-14 &    31,118,080 &  2,080,073    &  282,497  & 71.08\%  \\           
PCB 0.3nM & P0.3-16 &    30,420,319 &  1,940,661    &  291,059  & 69.71\%  \\           
  \hline                                                                             
PCB 10nM & P10-17  &    32,027,515 &  1,906,940    &  261,658  & 69.45\%  \\           
PCB 10nM & P10-18  &    30,103,549 &  1,826,707    &  237,354  & 70.88\%  \\           
PCB 10nM & P10-21  &    31,013,858 &  1,953,848    &  253,455  & 71.06\%  \\           
PCB 10nM & P10-23  &    32,226,790 &  2,043,756    &  276,857  & 70.03\%  \\           
PCB 10nM & P10-24  &    31,731,766 &  1,902,211    &  261,037  & 70.25\%  \\           
\hline
Average&         &    31,262,767 &  1,963,925    &  263,218  & 70,75\%  \\
  \end{tabular}
  \caption{Bisulfite statistics}
    \label{tab:methyl-stats}
\end{table}


\subsection{DNA methylation profiling}
General statistics, see Table \ref{tab:mapping-stats} and Table \ref{tab:methyl-stats}.


\subsection{PCB126-induced Changes in DNA Methylation in Testis}

\subsection{Very short methods}
BAT\_toolkit with metilene for DMR detection. GREAT for finding
corresponding genes and GO analysis.

\subsection{PCB126 0.3 nM treatment}
There is a total of 37 DMRs with 10 hypermethylated and 27
hypomethylated regions. (Two of the hypomethylated ones are on KN*
scaffolds and are therefore not covered by the annotations used later.)
None of them showed a percent methylation difference of larger than
40\%. Three hypomethylated DMRs had a methylation difference larger
than 30\%. The highest concentration of DMRs is on chromosome 4 with 9
DMRs, 24\% of the total amount. 7 of the 9 DMRs are hypomethylated.

Hypermethylated DMRs are significantly enriched for 17 GO molecular
function terms. The three most significant ones beeing GO:0008502
melatonin receptor activity, GO:0005242 inward rectifier potassium
channel activity and GO:0005249 voltage-gated potassium channel
activity. For biological processes there are 46 terms enriched,
e.g. GO:0051876 pigment granule dispersal, GO:0051877 pigment granule
aggregation in cell center, GO:0051905 establishment of pigment
granule localization.

Hypomethylated DMRs are enriched in 42 molecular function terms. The
most significant ones beeing GO:0003774 motor activity, GO:0008528
G-protein coupled peptide receptor activity, GO:0001653 peptide
receptor activity. For biological processes GO:0006397 mRNA
processing, GO:0042310 vasoconstriction and GO:0009132 nucleoside
diphosphate metabolic process are the most significant out of 29
terms.

\subsection{PCB126 10 nM treatment}
There is a total of 92 DMRs with 80 hypomethylated and 12
hypermethylated regions. (Six of the hypomethylated ones are on KN*
scaffolds and are therefore not covered by the annotations used later.)
13\% hypomethylated and 17\% hypermethylated show a percent
methylation difference larger than 40\%. The highest concentration of
DMRs is on chromosome 4 with 34 DMRs, 37\% of the total amount. 31 of
the 34 DMRs are hypomethylated.

Hypermethylated DMRs in the testis showed significant enrichment of a
number of GO molecular function terms. There are 42 in total, the
three most significant ones beeing GO:0070851 growth factor receptor
binding, GO:0008094 DNA-dependent ATPase activity, GO:0005343 organic
acid:sodium symporter activity. For biological processes there are 79
terms, e.g. GO:0006753 nucleoside phosphate metabolic process,
GO:0007166 cell surface receptor signaling pathway, GO:0044700 single
organism signaling.

Hypomethylated DMRs are enriched in 36 molecular function terms. The
most significant ones are GO:0043167 ion binding; GO:0003676 nucleic
acid binding and GO:0046872 metal ion binding. For biological
processes GO:0043631 RNA polyadenylation; GO:0016226 iron-sulfur
cluster assembly; GO:0016338 calcium-independent cell-cell adhesion
are the most significant of 34 different GO terms.
%The full list of GO terms is in the file %\TODO{Testis\_DMR\_GO\_analysis.xlsx}.

\subsection{PCB126-induced Transcriptional Changes in Testis}

\subsubsection{Very short methods}
DESeq2 with standard parameters. DAVID (https://david.ncifcrf.gov/)
for GO analysis.

\subsubsection{Results}
We obtained an average of 26.1 million reads mapping to ENSEMBL genes
in the DMSO-treated control sample. The libraries of individuals
treated with PCB126 0.3 nM or 10 nM resulted in 26.6 and 23.1 million
reads.

The gene Cyp1a is upregulated in the PCB126 0.3 nM and 10 nM treatment
by a log2 fold change of 7.6 and 8.9 respectively. In both cases the
adjusted p-value is below 0.05.

There were a total of 767 and 4,708 DEGs in the 0.3 nM and 10 nM
treatment with an adjusted p-value $<0.5$. Among the 767 DEGs in the
0.3 nM treatment, 458 were upregulated and 309 were
downregulated. Among the upregulated genes 214 (46.7\%) and in the
downregulated genes 144 (46.6\%), as well, had a fold change of more
than 2.

\begin{table}
  \begin{tabular}{ll}
    \hline
    {\bf Biological Process} - Upregulated & \\
    Term & Adj. p-value \\
    \hline
    GO:0009605~response to external stimulus & 0.0002 \\
    GO:0042330~taxis  & 0.0028 \\
    GO:0042221~response to chemical & 0.0024 \\
    GO:0006950~response to stress & 0.0155 \\
    GO:0006955~immune response & 0.0367 \\
    \hline
    {\bf KEGG} - Downregulated \\
    Term & Adj. p-value \\
    \hline
    dre04512:ECM-receptor interaction & 0.0085 \\
    dre04350:TGF-beta signaling pathway & 0.0050 \\    
  \end{tabular}
  \caption{GO terms of the PCB126 0.3 nM treatment. GOTERM\_BP\_2,
    GOTERM\_MF\_2 and KEGG pathway. Only the five best significant
    ones (adj. p-value $<$0.05) are shown. MF and KEGG for upregulated
    as well as BP and MF for downregulated genes had no significant
    enrichments.}
  \label{tab:go-0.3}
\end{table}

The upregulated genes were enriched in GO terms such as response to
external stimulus and taxis (both biological process). The
downregulated genes were only enriched in two KEGG pathways
ECM-receptor interaction and TGF-beta signaling pathway, see Table
\ref{tab:go-0.3} for details.

The PCB126 10 nM exposure resulted in the differential expression of
4,708 genes. Among these 2,822 genes were upregulated and 1,886 genes
were downregulated. Among the upregulated genes 1,534 (54.4\%) and in the
downregulated genes 324 (17.2\%) had a fold change of more than 2.

\begin{table}
  \begin{tabular}{ll}
    \hline
    {\bf Biological Process} - Upregulated & \\
    Term & Adj. p-value \\
    \hline
    GO:0006955~immune response & $<0.0001$ \\
    GO:0009605~response to external stimulus & $<0.0001$ \\
    GO:0050900~leukocyte migration & $<0.0001$ \\
    GO:0042330~taxis & $<0.0001$ \\
    GO:0042221~response to chemical & $<0.0001$ \\
    \hline
    {\bf Molecular Function} - Upregulated & \\
    Term & Adj. p-value \\
    \hline
    GO:0016491~oxidoreductase activity       &  $<0.0001$ \\
    GO:0004129~cytochrome-c oxidase activity & 0.0015 \\
    GO:0005515~protein binding               & 0.0052 \\
    GO:0030246~carbohydrate binding          & 0.0469 \\
    GO:0030234~enzyme regulator activity     & 0.0488 \\
     \hline
    {\bf KEGG} - Downregulated \\
    Term & Adj. p-value \\
    \hline
    dre00190:Oxidative phosphorylation  & $<0.0001$ \\
    dre04060:Cytokine-cytokine receptor interaction & $<0.0001$ \\
    dre04630:Jak-STAT signaling pathway & 0.0008 \\
  \end{tabular}
  \caption{GO terms of the PCB126 10 nM treatment. GOTERM\_BP\_2,
    GOTERM\_MF\_2 and KEGG pathway. Only the five best significant
    (adj. p-value $<$0.5) ones are shown. Downregulated genes had no
    significant enrichments.}
  \label{tab:go-10}
\end{table}

The upregulated genes of the PCB126 10 nM treatment are enriched in similar GO terms compared to the lighter PCB126 treatment, For example immune response and response to external stimulus. But opposite to the 0.3 nM treatment they are also enriched in molecular function GO terms, e.g. oxidoreductase activity and cytochrome-c oxidase activity but not in KEGG pathways. Downregulated genes of the 10 nM treatment were only enriched in three KEGG pathways Oxidative phosphorylation, Cytokine-cytokine receptor interaction and Jak-STAT signaling pathway, see Table \ref{tab:go-10} for details.

%\TODO{Make Supplement file for this part}

\subsection{Relationship between DMRs and Altered Gene Expression}

\subsection{Very short methods}
DMRs defined using metilene, corresponding gene using
GREAT. Subsequently, they were searched in differentially expressed
genes in the DESeq2 output.

\subsection{PCB126 0.3 nM treatment}
One hypermethylated DMR corresponds to an upregulated gene while three
hypomethylated DMRs correspond to two upd and one down-regulated gene,
for details see Table \ref{tab:correlation-0.3} (Please note: the DMR
IDs between both treatments are not the same).

\begin{table}
  \begin{tabular}{llll}
    DMR ID & Gene ID & Difference of mean  & log2 fold change\\
           &         & methylation rates per group & of Gene Expression \\
    \hline
    DMR\_3 & ENSDARG00000028661 & 0.14 & 0.78\\
    \hline
    DMR\_14 & ENSDARG00000103318 & -0.34 & 0.39\\
    DMR\_15 & ENSDARG00000103318 & -0.22 & 0.39\\
    DMR\_2 & ENSDARG00000052037 & -0.20 & -3.56\\
  \end{tabular}
  \caption{DMRs with DEGs and methylation difference as well as
    expression log2 fold change.}
  \label{tab:correlation-0.3}
\end{table}

\subsection{PCB126 10 nM treatment}
Two hypermethylated DMRs have an corresponding gene which is
significiantly upregulated. In both cases the gene is
downregulated. 15 hypomethylated DMRs correspond to an differentially
regulated gene. 12 of these genes are upregulated while 3 are
downregulated. The fold change ranges from 1.3 up to 8.6, for details
see Table \ref{tab:correlation-10} (Please note: the DMR IDs between
both treatments are not the same).

\begin{table}
  \begin{tabular}{llll}
    DMR ID & Gene ID & Difference of mean  & log2 fold change\\
           &         & methylation rates per group & of Gene Expression \\
    \hline
    DMR\_2 & ENSDARG00000030289 & 0.54 & -0.45\\
    DMR\_12 & ENSDARG00000005482 & 0.22 & -0.47\\
    \hline
    DMR\_6 & ENSDARG00000005185 & -0.24 & 3.10\\
    DMR\_34 & ENSDARG00000070845 & -0.18 & 1.89\\
    DMR\_70 & ENSDARG00000070845 & -0.18 & 1.89\\
    DMR\_35 & ENSDARG00000069311 & -0.20 & 1.51\\
    DMR\_69 & ENSDARG00000089382 & -0.20 & 1.21\\
    DMR\_10 & ENSDARG00000015472 & -0.23 & 1.16\\
    DMR\_1 & ENSDARG00000069996 & -0.40 & 0.90\\
    DMR\_3 & ENSDARG00000052361 & -0.13 & 0.87\\
    DMR\_11 & ENSDARG00000102824 & -0.21 & 0.75\\
    DMR\_73 & ENSDARG00000036567 & -0.21 & 0.61\\
    DMR\_17 & ENSDARG00000103318 & -0.16 & 0.41\\
    DMR\_18 & ENSDARG00000103318 & -0.47 & 0.41\\
    DMR\_22 & ENSDARG00000020730 & -0.20 & -0.55\\
    DMR\_22 & ENSDARG00000044718 & -0.20 & -0.58\\
    DMR\_30 & ENSDARG00000013312 & -0.20 & -0.60\\    
  \end{tabular}
  \caption{DMRs with DEGs and methylation difference as well as
    expression log2 fold change.}
  \label{tab:correlation-10}
\end{table}

\subsection{Problem with Gene Annotation}
For some genes the gene symbol used by GREAT can not be found in our
annotation file 'GRCz10.90.genes.gff'. Sometimes they can be
identified as ``Previous names'' in ZFIN but not always.

PCB126 0.3 nM treatment, hypermethylated DMRs: 2 gene names edited,
one not found (zgc:110222). Hypomethylated DMRs 2 gene names edited,
six not found (zgc:175264, si:dkey-63b1.1, chst13, trim63, cxcr7b,
pc).

PCB126 10 nM treatment, hypermethylated DMRs: 1 gene name edited;
onenot found (maf).  Hypomethylated DMRs 4 gene names edited, 18 genes
not found (si:ch211-202h22.10; tsc2; zgc:175264; si:dkey-63b1.1;
zgc:112437; si:dkey-4c15.4; si:dkey-51d8.7; si:dkey-78o7.1;
si:dkey-77p23.2; zgc:123060; si:dkey-45d16.8; si:dkey-10p5.9; pc;
rab35; zgc:112433; rhoad; slc16a1; tns1).

\TODO{Should we try other ways to find corresponding gene names?}


\end{document}